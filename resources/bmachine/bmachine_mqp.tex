\documentclass[a4paper,12pt]{report}
\author{Christian Bryan, Greg Opperman, Drew Wilson}
\date{\today}
\title{Broadcast Machine}
\usepackage{verbatim}

%% Things that need to be in this paper by the end of B term
%%
%%  * The background section of the paper should be finished. This includes:
%%    * The importance of open media and the importance of an open internet
%%      for open media (Greg)
%%    * Brief history of the software and PCF (Greg)
%%    * Reviews of similar software including Ruby, Rails, Word Press,
%%      Drupal/CivicSpace. (Simon for Wordpress - CJ for the rest)
%%    * Why we're building this on PHP/SQL (Done - See Requirements)
%%    * Explanations of the technologies we plan to use including HTTP,
%%      PHP, BitTorrent, Apache, SQL, and RSS (CJ?)
%%  * An overview of the existing software, how it's architected, and
%%    what's wrong with it. Include comments about MVC and the existing
%%    documentation (CJ)
%%  * A very detailed explanation of the existing data structures. This is
%%    vital for understanding the program. I've done most of the difficult
%%    work. (Greg)
%%  * A loose set of requirements. Having the existing data structures will
%%    give you most of this. (Greg)
%%  * A database schema that matches those requirements with a detailed
%%    prose explanation. We expect this will change a little. (Greg)
%%  * A plan for architecting the software. We expect this will change a
%%    little.
%%  * Enough of a start on the code to give us a proof-of-concept for the
%%    architecture
%%  * Unit tests for that code that integrate into the Democracy PyUnit test
%%    framework

\begin{document}

\maketitle

\tableofcontents

\chapter{Background}

\begin{comment}
* Importance of open media and an open internet for open media.
	* A brief history of the software and PCF.
	* How Broadcast Machine relates to similar software such as:
		> WordPress
		> Drupal
		> YouTube
	* Why Broadcast Machine will be built in PHP/MySQL versus Ruby on Rails, J2EE, and other PHP frameworks.
\end{comment}
\section{About the Participatory Culture Foundation}
	The Participatory Culture Foundation is a 501c(3) non-profit organization dedicated to the creation of a democratic mass medium through internet technology. 
Based in Massachussetts, it was founded by Worcester natives Nick Reville, Tiffiniy Cheng, Holmes Wilson, and WPI alunmus Nick Nassar.

	PCF seeks to achieve its goal of democratizing the media through their open-source Internet TV platform, consisting of several pieces.  Democracy Player allows users to download and watch videos from all over the world via RSS channels. 
Still in beta, Democracy Player has been rapidly expanding it's user base as it draws closer to a 1.0 release. 
In the month of November 2006 alone, the application was downloaded over 15,000 times.

Integrated seamlessly with Democracy, the Channel Guide serves as an open directory for internet television channels, ranging from local citizen journalism to major network programming. 
Any video publisher may submit their show to the Channel Guide, which users can then peruse by category from within Democracy.

Videobomb is a website for aggregating the best videos online, allowing users to create custom channels of videos from all over the web. 
Users post content that they think is noteworthy. 
Other users may "Star" the video, which bookmarks it as a favorite. 
The newest and most popular videos are aggregated onto the site's front page, or downloaded via feeds that integrate with Democracy Player.



\section{What is Broadcast Machine?}
	The final piece of the platform is Broadcast Machine, a tool that allows video producers to self-publish their content online.
	Broadcast Machine  began as Blog Torrent, a small application that simplified the creation and sharing of torrent files, as well as acted as a light-weight bitTorrent download client. 
Broadcast Machine built on that foundation, expanding the application to a full-featured toolkit for internet video publishers (commonly referred to as vloggers).
In its current state, it not only handles video distribution via bitTorrent, but also supports direct downloads, and creation of video channels compatible with Democracy (via RSS), albeit with problems.

\chapter{What is BitTorrent?}
%Should info like this be included as an appendix?
\section{Overview}
BitTorrent is a protocol for distributing large amounts of data across a network. 
Decentralized in design, The BitTorrent protocol allows individuals with limited upstream bandwidth to easily distribute files, minimizing the server load by sharing the bandwidth costs associated with file sharing. 

Currently the most popular form of filesharing, BitTorrent has changed the way that people share files on the Internet, making filesharing more democratic and accessible to ordinary users.
According to research conducted by Terry Shaw of CableLabs, and Jim Martin, a computer science professor at Clemson University, 60 percent of American teens equipped with broadband have downloaded audio and video files over peer-to-peer networks. 
In the same research paper, Shaw and Martin report that they found that 18\% of all broadband traffic, both upstream and downstream, is used for torrenting . 
According to their research, 55\% of broadband upstream (ie “home outward”) is used for torrenting. 
According to CacheLogic, a British web-analysis company, BitTorrent accounts for only 35 percent of traffic on the Internet, which is still more than all other peer-to-peer programs combined.  
In either case, BitTorrent accounts for a huge percentage of Internet traffic.

Bit Torrent has been criticized for enabling users to infringe upon copyright laws by sharing files illegally. Some Universities, including WPI, have gone to extreme measures to block access to the Bit Torrent protocol, despite its wide variety of legitimate uses. 


\section{Why is Bit Torrent Important?}
%This section sucks. Simon, I know you can do better. Write something the San-man would be proud of - Greg
There are thousands of uses for Bit Torrent. 
One of the most important is citizen journalism, and other efforts to democratize the media. 
Linus Torvalds, the father of Linux, said, “given enough eyeballs, all bugs are shallow”. 
That same idea can be applied to the media. 
With enough people watching, commenting and contributing, truth will emerge. It’s harder to deceive a well-informed public. 
Decentralization and participation are at the core of the Internet. 
Bulletin Board Systems (BBS), Usenet, email, chat-rooms, forums, blogs, podcasts, there’s a clear movement towards media participation. 
Internet video is the future of media and Bit Torrent is the equalizer that allows individuals to compete with big media companies at providing content. If a website uses Bit Torrent to distribute its content, it doesn’t need to spend millions of dollars on bandwidth to reach its viewers. With just a 10/month commercial web-hosting plan, a video camera, and PC, one person can reach millions of people.

\section{The Process}
To begin sharing files using the Bit Torrent protocol, users first need to install a Bit Torrent client. 
Popular clients include ABC, Azureus, BitTornado, TorrentStorm, and µTorrent. The same client is used to both upload and download Bit Torrent files.
The original host of the file, called the seeder, starts uses a Bit Torrent client to hash the file into hundreds of small pieces. 
Meta-data about the file is stored in a torrent file with the .torrent extension. 
Meta-data stored in this file includes, information about the files to be shared, how the data is cut up, and information about the host computer that will be seeding the files. 
This torrent file is then distributed to the people who are interested in downloading the data and is opened by a Bit Torrent client located on their personal computer. 
The torrent file can be distributed in many ways. 
Torrent files are very small, only a few kilobytes. They can be attached to an email, put on a website, posted on forums, etc. 
In fact, a new type of web-software has emerged called a torrent tracker.
Torrent trackers helps organize many torrent files into a single website. These websites generally include even more metadata than the data than the torrent files, such as the files title, its author, a website associated with the file, perhaps a description or review of the file. 
They also provide a the ability to search for a specific file. ChomskyTorrents.org is an example of a public torrent tracker.
Once a user opens a torrent file with their client, they begins downloading pieces of that file. 
They are now called a peer. 
The collection of peers downloading the file is called the swarm. 
In the beginning there is only one seeder in the swarm. 
The seeder starts by sends different pieces of the file to different peers. Once those peers finish downloading their first piece, they’re reading to share that piece with the rest of the swarm. 
So immediately after the first piece of the file is finished downloading, there is already another person ready to share the upstream of the file. 
The seeder and the peers connect to one another sharing pieces of the file until everyone in the swarm has every piece of the file. 
If one peer finishes the entire file before the others, eg if he or she has a faster downstream, that peer then also becomes a seeder, filling the gaps for the other peers.

\section{An Example}
Let’s say that a woman in Oaxaca, MX films the Mexican Federal Police attacking teachers on strike for better pay. 
She then creates a video file called HumanRightsAbusesInOaxaca-Nov162006.avi. She opens her Bit Torrent client and generates a torrent file. 
A new file called, HumanRightsAbusesInOaxaca-Nov162006.avi.torrent is created that keeps track of how the video file is “cut up”. 
She begins seeding the file and sends the torrent file to 100 of her contacts in the United States. 
These 100 people download the torrent file and open it in their Bit Torrent client. 
Their Bit Torrent client now has a list of all the pieces of the file and can begin requesting them all. 
The computer in Mexico begins by sending the first piece of the video file to Contact #1, the second piece of the video file to Contact #2, the third piece to Contact #3, and so on. 
Once some contact finishing downloading their piece of the video file, he or she can then share that piece with every other peer in the swarm. 
So now the computer in Mexico, the seeder, can concentrate on only sharing pieces of the file that only it has. 
Immediately every client in the swarm, ie every contact in the US, is both uploading and downloading the video file. 
The load is distributed among every client in the swarm. 
Furthermore, once the person in Mexico has sent 1 copy of every piece to some person in the swarm, the swarm can finish sharing the file by itself. 
Keeping the original seeder in the swarm will speed up the process, but its not necessary as long as every piece of the file exists somewhere in the swarm. 
For example, if the original seeder loses her connection to the Internet, the file can still be shared and distributed among the swarm. 

\chapter{Current Issues}

\begin{comment}
	Before we can go ahead and create a better Broadcast Machine, our team needs to understand the problems with the current software. While Broadcast Machine is good enough to meet the needs of most users, the goal of bringing an easy-to-install and trouble free means of distributing video is far from a reality. The problems are certainly not quick fixes either. Many of the complaints from current Broadcast Machine users are solvable, but only with hacks. The hacks themselves become hard to implement as the developer faces an architecture that is confusing at best. To figure out what the specific problems are, we split them into two categories. The first category of problems includes those that were caused by the specific implementation and the second involves those that spawned from a poorly designed architecture.
	
	One of the most useful features in Broadcast Machine is unfortunately the one that seems to cause the most problems. The BitTorrent distribution mechanism would allow users to distribute their video and not have to worry about the cost of bandwidth. This is at the center of a program like Broadcast Machine because its draw comes from people who lack the money to support complex distribution schemes for their content. The few times that it does work, the users have been happy with its performance, but its inconsistency is a huge drawback. While the BitTorrent mechanism’s instability is something that can be fixed, the server-side issues are much more complex. As many of the users are relatively inexperienced, and don’t have the money for more expensive webhosts, Broadcast Machine is often run on cheap servers with few options enabled. Trying to work around the limitations of many of these server configurations is a nightmare, but it’s necessary for much of the special functionality that Broadcast Machine has to offer.
	
	  When it comes time to fix bugs within Broadcast Machine, it’s certainly not smooth sailing for the programmer. For dealing with the database and the structure of the Model, it’s becomes difficult to tell what information is being stored in flat files, which data is being sent to the SQL database and how the two are switched and interconnected. This is where problems with the architecture of the program become a nuisance to the developers. When it comes time to fix the bugs, it’s difficult to do so because the different functions of the program aren’t separated at all in places and where they are, the implementation is lacking. 

	Another one of Broadcast Machine’s failings is in the documentation department. With an open source project like this which can possibly thrive on the contributions of users and other developers, documentation is critical. When tech savvy users have a problem and the desire to fix it, the Broadcast Machine team leaves them out to dry. For a stranger to start working on this program with little face-to-face explanation would be quite the feat. While this problem is less technical in nature, it should be a large concern to the maintainers of Broadcast Machine. Prospective developers are choosing not to hack on the project when they realize that it is often not worth their time to deconstruct the code. 
\end{comment}

% Work in progress! -CJ
	Before creating a better Broadcast Machine, we need to understand its current shortcomings. 
The first step in addressing problems is understanding what they are. 
There are a few issues that are often brought up by users and a few more that the developers are concerned about. 
While the current Broadcast Machine package is working for over 20,000 users (2.1) there are significant shortcomings with the software's current architecture and documentation.

% Incomplete -CJ
	Many successful applications, including Broadcast Machine, are architected in such a way that the program is split up into three distinct parts. 
The first is the Model which describes how the data in the program is stored, the second is the View which is responsible for how the program interacts with the end user and the third is the Controller which bridges the gap between Model and View by responding to events raised by the other components.

% Incomplete -CJ
	In Broadcast Machine, this architecture is present but its implementation is lacking. 
The most notable example of this is the data store. 
The data store is described in datastore.php and is built to supply the rest of the program with a means of setting and retrieving data for the program. In this respect, the data store is a success, but some of the functionality that should be in the data store is spread out in other parts of the application, while some aspects that should be abstracted to the controller layer are present.
	
% Complete! But I feel like this could be split into two paragraphs. -CJ	
	The second area in which Broadcast Machine is lacking is documentation. 
While there is enough user end documentation to support the existing community there is certainly room for improvement. 
The walkthroughs are often only a paragraph long and describe a bare minimum of features available in Broadcast Machine. 
The occasional screenshot helps out by showing the user where to navigate, but this is unfortunately in the minority of cases. 
Supporting the end user is certainly an issue for a project that wishes to put powerful software in the hands of users, but even more alarming is the sparse developer documentation. 
An open-source project like Broadcast Machine can and should leverage the expertise of the community to build a better product and the lack of API documentation and comments in the code only make it more difficult for other developers to improve upon the software. 
There are a few files that begin to talk about the data structures used in the program, but it stops short. 
While it is certainly a good start, it needs to be cleaned up and completed. There is a certain amount of inline documentation as well, but it is far from complete. 
The comments that are there describe what the variables are there for, but fail to mention how the a function might work or what effects it has on other parts of the application.
	
% There should be a conclusion paragraph here. -CJ

(2.1) A search for "Powered by Broadcast Machine" on Google returns upwards of 20,000 results.
http://www.google.com/search?q=%22Powered+by+Broadcast+Machine%22

\chapter{Requirements and Design Decisions}

\section{Architecture}

	Many of Broadcast Machine’s current problems stem from its poor architectural implementation. 
To prevent these problems from happening in the future, Broadcast Machine must have a carefully planned architecture. 
From a design standpoint, a successful architecture has several requirements.

The application must be designed so that a user with only basic HTML and CSS knowledge can easily customize the website to suit his or her needs. 
Ideally, users’ site layouts, or “themes”, have the potential to look drastically different. 
The templates must be abstracted from the functional code, so that users needn’t worry about damaging the code or digging deep into its inner workings when modifying the layout. 
Users should also have the ability to easily switch between layouts.

Along the same lines, the code must be structured so that it is easily maintainable and extendable by any developer. 
The layout of the application should appear logical and concise. 
In the event that bugs occur, localizing them to a specific section of code and implementing a fix should be possible without the fix appearing hacked together or thrown in. 
Developers should not need to dig through a mountain of code before finding the section that they need to edit. 
If features need to be added, the developers should be able to do so while maintaining the same architectural pattern and preserving the structure of the application. 
In this sense, a good architecture will encourage good design practices.

The application must be flexible enough so that it can be re-factored easily. Functionality should be abstracted and delegated so that major changes to one part of the code do not affect the others. 
For example, we may decide later that we would like to use a different type of database, or even a flat file system. We should be able to swap out the back end without having to modify the entire application, and without significantly affecting the user experience.

For these reasons, we have chosen a Model-View-Controller design pattern to represent our application. 
Model-View-Controller (abbreviated MVC) separates the application into three distinctive layers with different responsibilities that interact to comprise the program. 
The View layer covers any part of the application that the user interacts with, such as the actual web page that the program renders. 
All of the actual data stored by the application is represented by the Model layer. 
The Controller layer handles interactions between the View and the Model, and is responsible for invoking changes on either layer.

\section{Compatibility}

The new Broadcast Machine must work across several different platforms. 
While impossible to guarantee compatibility with all web server configurations, Broadcast Machine most work on the most common web server setups. 
This includes any Apache web server with a minimal amount of installed modules, and most major hosting services (Dreamhost, 1And1, etc). 

For this reason, we decided that the program’s front end should be written in PHP, while the back end data layer will be represented by a MySQL database. Although Ruby on Rails is ideal for a Model-View-Controller web application (as it enforces the architecture by its nature, and automatically generates and maintains basic interactions between the layers), it hasn’t been fully adopted by most of the web community. 
Very few hosting services include Ruby in their hosting plan, and setting up Rails manually is a task beyond our target user-base. 
PHP, while a less than perfect programming language, comes pre-configured with most Apache installs, and is available with most, if not all, web hosting plans. 

Similarly, MySQL is an open-source DBMS (Database Management System) that is perhaps the most widely used SQL (Structured Query Language) implementation, with the exception of enterprise-level applications such as Oracle. 
Like PHP, MySQL comes installed by default with most hosting plans, and setting up a database to be used with our application is a matter of running a script with the correct permissions. 

When choosing our DBMS, we must also predict how the application will scale with databases of varying sizes. 
An assumption that the average database will contain less than 1000 records for unique videos is reasonable. A publisher who produces a show daily would take almost three years to reach 1000 records, and most publishers release videos somewhat less frequently. 
Most of the queries on the database will be basic select statements for getting data, and insert or alter statements for adding or modifying video information. 
The predicted scale and function of our database would be best suited for a minimalist DBMS such as SQLite, which optimizes access to the data layer for smaller databases that do not need advanced functionality. 
Ideally, the program will be flexible enough to use SQLite when available, and require MySQL in all other instances. 
Architecting a flexible database controller will allow us to add other DBMS choices as necessary.

\section{Ease of Use}

	The most obvious requirement for Broadcast Machine is that it be easy to use. 
Our target user, people who produce videos, should not need a wealth of technical knowledge in order to deploy and use the program. 
Just as we have almost no knowledge of how to shoot or edit a video, we should not expect the user to have any knowledge of PHP, RSS, or any of the technical concepts employed. 
Without being privy to any of the inner workings of the program, the user must be able to easily and intuitively publish videos and video “channels” (a simplified term for democracy-compatible RSS syndication) via a simple administrative interface. 
Without any information besides their web server login and password, the user should be able to set up Broadcast machine by dropping the application in their public html folder. 
The program should auto-configure itself on the first-run with a few simple clicks, even if the user is completely unaware of how his or her web server is set up. 
The application must intuitively detect server configurations that may interfere with its behavior, and handle those special cases gracefully.

Likewise, viewers with any level of computer proficiency must be able to easily browse and download videos, leave comments and feedback, and subscribe to their favorite channels in Democracy Player (See use cases for more information).

\section{Stability}

	Most importantly, the new Broadcast Machine must be stable. 
At the very least, it should preserve all of the functionality of its previous incarnation, while containing none of the bugginess or unpredictable behavior. 

\chapter{Architecture}

\section{View}

\section{Controller}

\section{Model}
Broadcast Machine must keep track of  several pieces of information pertaining to its entities (channels, files, etc.). 
A comprehensive list of the data follows (entity relationships highlighted in bold):

\subsection{Data associated with Channels}
Channel ID – An arbitrary, unique number to identify the channel and  associate it with other entities
Title – Each Channel has a title, represented by a string
Description - A longer body of text that describes the channel's theme or content
Icon – A thumbnail image for the channel
Credits – A list of the channel creators and maintainers, or any other people involved. 
Donation information – Represented as a snippet of HTML or a simple url.
Website information – A url to link to the channel's main web page.
Tags – 0 or more keywords associated with the channel, used for categorizing and browsing multiple channels.
Permissions – Rules about which users can access or modify the channel

\subsection{Data associated with channel tags:}
Video ID – A reference to the video that the tag describes
Tag name – A short word or phrase to categorize the channel

\subsection{Data associated with Users:}
ID – An arbitrary, unique number to identify the user and associate him/her with other entities
Username – The user's nickname or handle
Password – Self-explanatory
Email Address – Optionally used to verify registration
Permissions – User's access credentials for specific channels, files, and the site in general

\subsection{Videos have:}
ID – An arbitrary, unique number to identify the user and associate it with other entities 
Title – A short amount of text to describe the video
Description - A longer body of text that describes the video's theme or content
Last Modified Date – A time stamp set every time any video information is changed.
Credits – A list of the channel creators and maintainers, or any other people involved. 
Icon – A thumbnail image for the video
Transcript (optional)– A text-version of the video's content or dialog for the hearing impaired
License/Copyright information – The name of the license associated with the content, and possibly a url to link to the full body of the license
Website url – A link to the video  or publisher's website
Release Date – The date and time the video was released to the public
Publish Date – A timestamp of when the video is made available to the public. note that this is different than release time (For example, a publisher might publish a movie that was released at an earlier time elsewhere).
Running time – The length of the video, in seconds
Adult – A boolean flag used for marking adult content
Donation information – Represented as a snippet of HTML or a simple url.
Tags – 0 or more keywords associated with the channel, used for categorizing and 	browsing multiple channels.
File information – Data about the actual video file itself (see below).
Permissions – Rules about which users can access or modify the video

\subsection{Data associated with video tags:}
	Video ID – A reference to the video that the tag describes
	Tag name – A short word or phrase to categorize the video

\subsection{Data associated with Files:}
ID – A reference to the ID number of the video to which the file belongs
Mime-type – The type of media that the file represents
Filename – the url or address on the server where the file is located
Size – the size of the file, in bytes.
Downloads – A running count of the number of users who have downloaded the file
BitTorrent Information (optional) – If the file is a torrent, additional information must be stored (see below)

\subsection{Data associated with Torrents:}
(To be completed)

(Should we pop the schema itself in here?)

\subsection{Design Decisions}
The current schema departs from the previous data structures in many ways. Previously, the data layer was represented as a flat-file database. 
Although most of the data structures were organized in a normalized, logical way, there were several inconsistencies that needed to be reconciled.

First, several pieces of data were being improperly stored in the database. For example, channel and site settings were being stored in the database. This information doesn't typically belong in a relational data structure, simply because it does not interact with other entities, nor can it benefit from any of the features of a database. 
Instead, we decided that the data belongs in a simple, transparent configuration file as part of the template system. 
That way, data can be easily read and modified, both by the application and advanced users. 

The new schema also drops several fields and data structures that could be simplified or generated on the fly. 
For example, in the previous version, thumbnails were represented as their own entities, because the database stored the mime-type as well as the file address. 
Since most, if not all, browsers render images without an explicit mime-type, eliminated this field also allows us to simplify the schema by storing all of the thumbnail information in the channels and video tables themselves. Similarly, fields like file extension, which can be derived from other data, have been pruned. 
We also store information about intellectual property licenses that could be abstracted to its own table. 
However, this doesn't provide any significant functional benefits, so the information has been moved to data fields in other tables.

Although file data and video data have a one-to-one relationship, we decided that it makes more logical sense to keep them separate rather than merge them into one table. 
Although we lose a small amount of database performance in a few cases (when file data and video data need to be aggregated together), the alternative would leave us with a large table that would be difficult to maintain. Breaking up the data into two logically chunked tables allows us to write simpler SQL statements when getting and setting, and make these statements easier to read and understand, which has unforeseen benefits in the event that the data structures need to be re-factored.

\end{document}
